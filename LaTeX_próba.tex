\documentclass{article}
\usepackage[MeX]{polski}
\usepackage[utf8]{inputenc}
\begin{document}
% Przykład 1
\ldots Słynne równanie Einsteina
\begin{equation}
e = m \cdot c^2 \; ,
\end{equation}
jest zarówno najbardziej znanym, ale także
najmniej rozumianym równaniem w~fizyce.
% Przykład 2
\ldots którego wynikiem jest prądowe prawo Kirchhoffa:
\begin{equation}
\sum_{k=1}{n} I_k = 0 \; .
\end{equation}
Natomiast, napięciowe prawo Kirchhoffa ma
swój początek w~\ldots
% Przykład 3
\ldots co ma określone zalety.
\begin{equation}
I_D = I_F - I_R
\end{equation}
jest rdzeniem innego modelu tranzystora. \ldots
\newline
\newline



Mr.~Smith was happy to see her\\
cf.~Fig.~5\\
I like BASIC\@. What about you?



Żeby dokument można było wyświetlić w poglądzie (PDF czy DVI)
przy zapisywaniu musimy ręcznie dodać końcówkę ".tex" do zapisywanego pliku.\newline
No chyba że lubisz przygody, to polecam udać się w ten las zwanym LaTeX\ldots
\newline
\newline
Jak masz czas i chęci to otwórz sobie ten "cud piśmienniczy" w owym programie
% NO I OCZYWIŚCIE ŻE TRZEBA ZAINSTALOWAĆ MiKTeX 
%NO I ZMIENIĆ USTAWIENIA W "KONFIGURACJA TATEX ŻEBY TO WYŚWIETLAŁO JAK NALEŻY.

\end{document}